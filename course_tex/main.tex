\documentclass[12pt]{article}
\usepackage[utf8]{inputenc}
\usepackage{enumerate}
\usepackage{fancyhdr}
\usepackage{times}
\usepackage[top=2cm, bottom=2cm, left=2cm, right=2cm]{geometry}


\pagestyle{fancy}
\fancyhf{}
\fancyhead[C]{Yapay Zeka Eğitimi}
\fancyfoot[L]{Muhammed Sezer \& Şevval B. Dikkaya}
\fancyfoot[R]{\thepage}

\title{Ders Yol Haritası}
\author{}
\date{}

\begin{document}

\maketitle
\thispagestyle{fancy}

\section{Ders Yol Haritası}

\begin{enumerate}
    \item Giriş (1 saat)
    \begin{itemize}
        \item Yapay zeka, makine öğrenimi ve derin öğrenme kavramlarının tanıtılması
        \begin{itemize}
            \item Yapay zeka: İnsan gibi düşünme ve öğrenme yeteneğine sahip bilgisayar sistemleri
            \item Makine öğrenimi: Yapay zekanın alt dalı olarak, veri ve deneyimle otomatik olarak öğrenen ve gelişen algoritmalar
            \item Derin öğrenme: Makine öğreniminin alt dalı olarak, büyük veri kümesinden karmaşık özellikler öğrenebilen ve hiyerarşik öğrenme sağlayan yapay sinir ağları temelli algoritmalar
        \end{itemize}
        \item Yapay zeka uygulama alanları ve örnekler
        \begin{itemize}
            \item Görüntü işleme: Görüntü sınıflandırma, nesne tanıma ve yüz tanıma gibi görsel problemler için derin öğrenme algoritmaları kullanılır.
            \item Doğal dil işleme: Metin analizi, duygu analizi ve metin sınıflandırma gibi dil temelli problemler için makine öğrenimi ve derin öğrenme algoritmaları kullanılır.
            \item Ses işleme: Ses tanıma, konuşma tanıma ve ses sınıflandırma gibi sesle ilgili problemler için derin öğrenme algoritmaları kullanılır.
            \item Oyunlar ve strateji: Satranç ve Go gibi oyunlarda en iyi hamleleri belirlemek için yapay zeka algoritmaları kullanılır.
            \item Tıp ve sağlık: Hastalık teşhisi, tıbbi görüntü analizi ve kişiselleştirilmiş tedavi önerileri için yapay zeka algoritmaları kullanılır.
            \item Otomotiv: Otonom araçlar ve sürüş destek sistemleri için yapay zeka ve derin öğrenme algoritmaları kullanılır.
        \end{itemize}
        \item Yapay zeka türleri ve faydaları
        \begin{itemize}
            \item Gözetimli öğrenme: Etiketli veri kullanarak öğrenen ve doğru tahminlerde bulunmayı amaçlayan algoritmalar. Özellikle sınıflandırma ve regresyon problemlerinde başarılıdır.
            \item Gözetimsiz öğrenme: Etiketli veri olmadan verinin yapısal özelliklerini öğrenen algoritmalar. Kümeleme, boyut azaltma ve veri yoğunluğu tahmini gibi problemlerde kullanılır.
            \item Takviyeli öğrenme: Belirli bir hedefe yönelik hareketler yaparak deneme yanılma yöntemiyle öğrenen algoritmalar. En iyi eylemleri seçerek ödül mekanizmasıyla çalışır. Oyunlar, robotik ve gerçek zamanlı karar verme problemlerinde kullanılır.
            \item Yarı-gözetimli öğrenme: Hem etiketli hem de etiketsiz veri kullanarak öğrenen algoritmalar. Büyük veri kümesinde küçük etiketli veri bulunan durumlarda başarılıdır.
            \item Aktarım öğrenimi: Önceden eğitilmiş bir modelin bilgisi ve ağırlıklarının, yeni ve benzer bir problem için kullanılması. Hızlı öğrenme ve daha az etiketli veri gereksinimi sağlar.
        \end{itemize}
        \item Genel kavramlar ve terminoloji
        \begin{itemize}
            \item Özellik mühendisliği: Algoritmanın öğrenmesi için veriyi uygun şekilde temsil etme ve ön işleme yapma süreci.
            \item Hiperparametre ayarlama: Yapay zeka modelinin performansını en üst düzeye çıkarmak için algoritmanın kontrol parametrelerini seçme ve ayarlama süreci.
            \item Model doğrulama ve değerlendirme: Eğitilmiş yapay zeka modelinin performansını ölçme ve değerlendirme süreci. K-Fold çapraz doğrulama, doğruluk, hatırlama ve F1 skoru gibi metrikler kullanılır.
        \end{itemize}
    \end{itemize}


    \item Regresyon (4 saat)
    \begin{itemize}
        \item Doğrusal regresyon ve lojistik regresyonun temelleri
        \begin{itemize}
            \item Doğrusal regresyon: Sürekli değerleri tahmin etmek için kullanılan bir regresyon türü
            \item Lojistik regresyon: İkili sınıflandırma problemleri için kullanılan bir regresyon türü
            \item Regresyon optimizasyonu: Maliyet fonksiyonunu minimize ederek regresyon modelinin ağırlıklarını ve önyargılarını ayarlama süreci
        \end{itemize}
        \item Regresyon modellerinde türev ve backpropagation
        \begin{itemize}
            \item Türev: Maliyet fonksiyonunun değişkenlere göre değişim hızı, optimizasyon için kullanılır
            \item Regresyon optimizasyonunda türevin rolü: Ağırlıkların ve önyargıların güncellenmesi için kullanılır
            \item Backpropagation: Derin sinir ağlarında hata ve türevlerin geriye doğru yayılmasıyla ağırlıkların güncellenmesi süreci, temel olarak regresyon optimizasyonuna dayanır
        \end{itemize}
        \item Pratik örnekler ve öğretmenlerin deneyebileceği basit regresyon projeleri
        \begin{itemize}
            \item Ev fiyatlarının tahmin edilmesi: Doğrusal regresyon kullanarak sürekli değerlerin tahmin edilmesi
            \item E-postaların spam olup olmadığının tespiti: Lojistik regresyon kullanarak ikili sınıflandırma problemlerinin çözümü
        \end{itemize}
        \item Python kullanarak regresyon modelleri oluşturma
        \begin{itemize}
            \item Optimizasyon algoritmaları ve hiperparametrelerin ayarlanması
            \item Regresyon modellerinin değerlendirilmesi ve performans metrikleri
        \end{itemize}
    \end{itemize}

    
    \item Derin Sinir Ağları (DNN) (4 saat)
    \begin{itemize}
        \item Yapay sinir ağları ve DNN temelleri
        \begin{itemize}
            \item Yapay sinir ağları: İnsan beyninin işleyişine benzer şekilde öğrenen ve tahminlerde bulunan algoritmalar
            \item Derin sinir ağları: Çok katmanlı yapay sinir ağları, karmaşık özellikler öğrenme yeteneği
            \item Katmanlar ve nöronlar: DNN'deki temel yapı taşları, her bir katman farklı özellikler öğrenir
        \end{itemize}
        \item Aktivasyon fonksiyonları, kayıp fonksiyonları ve optimizasyon teknikleri
        \begin{itemize}
            \item Aktivasyon fonksiyonları: ReLU, Sigmoid, Tanh gibi nöronların çıktılarını dönüştüren fonksiyonlar
            \item Kayıp fonksiyonları: Mean Squared Error, Cross-Entropy gibi DNN performansını ölçen ve optimizasyon için kullanılan fonksiyonlar
            \item Optimizasyon teknikleri: Stokastik Gradyan İnişi (SGD), Adam gibi ağırlıkların ve önyargıların güncellenmesi için kullanılan yöntemler
        \end{itemize}
        \item PyTorch kullanarak basit DNN modelleri oluşturma
        \begin{itemize}
            \item PyTorch ile DNN modelinin tanımlanması ve eğitilmesi
            \item Katmanlar, nöronlar ve aktivasyon fonksiyonlarının seçimi
            \item Modelin hiperparametrelerinin ayarlanması ve performansının değerlendirilmesi
        \end{itemize}
        \item Pratik örnekler ve öğretmenlerin deneyebileceği basit DNN projeleri
        \begin{itemize}
            \item El yazısı rakam tanıma: MNIST veri kümesi kullanarak derin sinir ağları ile görüntü sınıflandırma
            \item Metin sınıflandırma: Film yorumlarının duygu analizi için derin sinir ağları kullanma
            \item Ses tanıma: Basit komutları tanımak için derin sinir ağları kullanarak ses sınıflandırma
        \end{itemize}
    \end{itemize}

    
    \item Evrişimli Sinir Ağları (CNN) (3 saat)
    \begin{itemize}
        \item CNN temelleri ve görüntü sınıflandırma
        \begin{itemize}
            \item Evrişimli katmanlar: Görüntü verisinden özellikleri öğrenen ve filtreler kullanan katmanlar
            \item Havuzlama katmanları: Görüntü boyutunu küçültmek ve bilgi yoğunlaştırmak için kullanılan katmanlar
            \item Tam bağlantılı katmanlar: Öğrenilen özellikleri kullanarak sınıflandırma işlemini gerçekleştiren katmanlar
            \item Görüntü sınıflandırma: Evrişimli sinir ağları kullanarak görüntülerin etiketlerine göre sınıflandırılması
        \end{itemize}
        \item Öğretmenlerin deneyebileceği basit CNN projeleri
        \begin{itemize}
            \item Evrişimli sinir ağları ile daha karmaşık görüntü sınıflandırma problemleri (ör. CIFAR-10, CIFAR-100)
            \item Yüz tanıma: CNN'ler kullanarak yüzlerin tanınması ve kişilerin tespit edilmesi
            \item Görüntü segmentasyonu: CNN'ler kullanarak görüntülerin bölümlere ayrılması ve nesnelerin sınırlarının belirlenmesi
        \end{itemize}
        \item PyTorch kullanarak basit CNN modelleri oluşturma
        \begin{itemize}
            \item PyTorch ile CNN modelinin tanımlanması ve eğitilmesi
            \item Evrişimli, havuzlama ve tam bağlantılı katmanların seçimi ve kullanımı
            \item Modelin hiperparametrelerinin ayarlanması ve performansının değerlendirilmesi
        \end{itemize}
    \end{itemize}

    
    \item YOLO ve Nesne Algılama (4 saat)
    \begin{itemize}
        \item Nesne algılama ve YOLO algoritması hakkında bilgi
        \begin{itemize}
            \item Nesne algılama: Görüntülerdeki nesnelerin sınıflandırılması ve konumlarının belirlenmesi
            \item YOLO (You Only Look Once): Hızlı ve gerçek zamanlı nesne algılama için geliştirilen bir CNN mimarisi
            \item YOLO'nun avantajları: Hızlı çalışma süresi, doğru tahminler ve gerçek zamanlı uygulamalara uygunluk
        \end{itemize}
        \item PyTorch ve YOLOv8 kullanarak nesne algılama uygulamaları
        \begin{itemize}
            \item YOLOv8 modelinin PyTorch ile yüklenmesi ve kullanılması
            \item Önceden eğitilmiş YOLOv8 modelini kullanarak nesne algılama gerçekleştirme
            \item YOLOv8 modelinin kendi veri kümesi üzerinde eğitilmesi ve ayarlanması
        \end{itemize}
        \item Öğretmenlerin deneyebileceği basit nesne algılama projeleri
        \begin{itemize}
            \item Gerçek zamanlı trafik işaretleri tanıma: YOLOv8 kullanarak trafik işaretlerinin tespit edilmesi ve sınıflandırılması
            \item Yüz maskesi tespiti: YOLOv8 kullanarak insanların yüz maskesi takıp takmadığının belirlenmesi
            \item Hayvan türlerinin sınıflandırılması: YOLOv8 ile doğadaki hayvanların tespit edilmesi ve türlerine göre sınıflandırılması
        \end{itemize}
    \end{itemize}

    
    \item Doğal Dil İşleme (NLP) (4 saat)
    \begin{itemize}
        \item NLP temelleri ve örnek uygulamalar
        \begin{itemize}
            \item Doğal Dil İşleme (NLP): İnsan diliyle ilgili bilgi işlem ve anlama problemlerini çözmeye yönelik alan
            \item NLP'nin alt disiplinleri: Dil modelleme, sözdizimi analizi, anlamsal analiz ve duygu analizi
            \item NLP uygulama örnekleri: Metin sınıflandırma, makine çevirisi, özetleme ve sohbet botları
        \end{itemize}
        \item Duygu analizi, metin sınıflandırma ve özetleme gibi temel NLP problemleri
        \begin{itemize}
            \item Duygu analizi: Metinlerdeki duyguları ve tonları tespit etme
            \item Metin sınıflandırma: Metinleri belirli kategorilere göre sınıflandırma
            \item Özetleme: Metinlerin ana fikirlerini kısa ve öz bir şekilde özetleme
        \end{itemize}
        \item Pratik NLP projeleri ve PyTorch kullanarak öğretmenlerin deneyebileceği basit NLP uygulamaları
        \begin{itemize}
            \item TODO
        \end{itemize}
    \end{itemize}

    
    \item Yapay Zeka Etik ve Sorumlulukları (2 saat)
    \begin{itemize}
        \item Yapay zeka etiği, önyargı ve veri gizliliği
        \begin{itemize}
            \item Algoritmik önyargı: Yapay zeka sistemlerinin veri ve algoritmalar nedeniyle önyargılı sonuçlar üretmesi
            \item Veri gizliliği: Kişisel verilerin korunması ve güvence altına alınması
            \item Karar verme süreçlerinin şeffaflığı: Yapay zeka sistemlerinin kararlarının açıklanabilir ve anlaşılır olması
        \end{itemize}
        \item Öğrencilere yapay zeka konularını öğretirken dikkate alınması gereken etik konular
        \begin{itemize}
            \item Öğrencilere yapay zeka sistemlerinin etik ve sosyal etkilerinin öğretilmesi
            \item Öğrencilere doğru ve etik veri kullanımı konusunda bilgi verilmesi
            \item Yapay zeka teknolojilerinin olumlu ve olumsuz potansiyel etkilerine dikkat çekmek
        \end{itemize}
    \end{itemize}
    
    \item Öğretmenler için Yapay Zeka Proje Fikirleri (2 saat)
    \begin{itemize}
        \item Öğretmenlerin kendi derslerinde uygulayabileceği yapay zeka proje önerileri
        \begin{itemize}
            \item Öğrencilere yönelik sınıf içi ve ödev projeleri
            \item Disiplinler arası yapay zeka projeleri: Yapay zeka tekniklerinin farklı derslerde kullanılması
            \item Yapay zeka ile ilgili etkinlikler ve yarışmalar düzenlemek
        \end{itemize}
        \item Kursun değerlendirilmesi ve geri bildirim alınması
        \begin{itemize}
            \item Öğretmenlerin kurs hakkındaki düşüncelerini ve önerilerini toplama
            \item Kursun başarısını değerlendirmek için anketler ve geri bildirim formları kullanma
            \item Gelecekte düzenlenecek eğitimler için öğretmenlerden alınan geri bildirimleri dikkate alma
        \end{itemize}
    \end{itemize}

    \item Yararlı Uygulama, Tool ve Websiteleri Tanıtımı (2 saat)
    \begin{itemize}
        \item Online yapay zeka ve makine öğrenimi dersleri: Coursera, Udacity, edX, fast.ai
        \item Yapay zeka ve makine öğrenimi kaynakları ve topluluklar: Machine Learning Mastery, Google AI Blog, AI Hub, Kaggle
        \item Yapay zeka ve makine öğrenimi araçları ve kütüphaneler: TensorFlow, Keras, Scikit-learn, OpenAI Gym, PyTorch, Netron, Ultralytics YOLOv8, NVIDIA Jetson Inference, ChatGPT, Google BERT
        \item Veri seti kaynakları: UCI Machine Learning Repository, Google Dataset Search, Data.gov, Kaggle Datasets
        \item Yapay zeka ve makine öğrenimi görselleştirme ve deney araçları: Google Colab, Jupyter Notebook, TensorBoard, MLflow
        \item Otomatik makine öğrenimi (AutoML) platformları: Google AutoML, H2O.ai, DataRobot, Microsoft Azure AutoML
        \item Yapay zeka ve makine öğrenimi modellerini dağıtmak için araçlar: TensorFlow Serving, MLflow Model Serving, NVIDIA Triton Inference Server, TorchServe
        \item Yapay zeka ve makine öğrenimi için önceden eğitilmiş modeller ve API'ler: TensorFlow Hub, Hugging Face Transformers, OpenAI API, IBM Watson API
        \item Yapay zeka ve makine öğrenimi güncellemeleri ve haberleri için kaynaklar: AI Weekly, Import AI, AI Podcasts, Arxiv Sanity
    \end{itemize}


\end{enumerate}

\end{document}
